
% Default to the notebook output style

    


% Inherit from the specified cell style.




    
\documentclass[article, floatfix, groupaddress, prb]{revtex4-1}

\usepackage{placeins}

\AtBeginDocument{
\heavyrulewidth=.08em
\lightrulewidth=.05em
\cmidrulewidth=.03em
\belowrulesep=.65ex
\belowbottomsep=0pt
\aboverulesep=.4ex
\abovetopsep=0pt
\cmidrulesep=\doublerulesep
\cmidrulekern=.5em
\defaultaddspace=.5em
}


    
    
    \usepackage[T1]{fontenc}
    % Nicer default font than Computer Modern for most use cases
    \usepackage{palatino}

    % Basic figure setup, for now with no caption control since it's done
    % automatically by Pandoc (which extracts ![](path) syntax from Markdown).
    \usepackage{graphicx}
    % We will generate all images so they have a width \maxwidth. This means
    % that they will get their normal width if they fit onto the page, but
    % are scaled down if they would overflow the margins.
    \makeatletter
    \def\maxwidth{\ifdim\Gin@nat@width>\linewidth\linewidth
    \else\Gin@nat@width\fi}
    \makeatother
    \let\Oldincludegraphics\includegraphics
    % Set max figure width to be 80% of text width, for now hardcoded.
    \renewcommand{\includegraphics}[1]{\Oldincludegraphics[width=.8\maxwidth]{#1}}
    % Ensure that by default, figures have no caption (until we provide a
    % proper Figure object with a Caption API and a way to capture that
    % in the conversion process - todo).
    \usepackage{caption}
    \DeclareCaptionLabelFormat{nolabel}{}
    \captionsetup{labelformat=nolabel}

    \usepackage{adjustbox} % Used to constrain images to a maximum size 
    \usepackage{xcolor} % Allow colors to be defined
    \usepackage{enumerate} % Needed for markdown enumerations to work
    \usepackage{geometry} % Used to adjust the document margins
    \usepackage{amsmath} % Equations
    \usepackage{amssymb} % Equations
    \usepackage{textcomp} % defines textquotesingle
    % Hack from http://tex.stackexchange.com/a/47451/13684:
    \AtBeginDocument{%
        \def\PYZsq{\textquotesingle}% Upright quotes in Pygmentized code
    }
    \usepackage{upquote} % Upright quotes for verbatim code
    \usepackage{eurosym} % defines \euro
    \usepackage[mathletters]{ucs} % Extended unicode (utf-8) support
    \usepackage[utf8x]{inputenc} % Allow utf-8 characters in the tex document
    \usepackage{fancyvrb} % verbatim replacement that allows latex
    \usepackage{grffile} % extends the file name processing of package graphics 
                         % to support a larger range 
    % The hyperref package gives us a pdf with properly built
    % internal navigation ('pdf bookmarks' for the table of contents,
    % internal cross-reference links, web links for URLs, etc.)
    \usepackage{hyperref}
    \usepackage{longtable} % longtable support required by pandoc >1.10
    \usepackage{booktabs}  % table support for pandoc > 1.12.2
    \usepackage[normalem]{ulem} % ulem is needed to support strikethroughs (\sout)
                                % normalem makes italics be italics, not underlines
    

    
    
    % Colors for the hyperref package
    \definecolor{urlcolor}{rgb}{0,.145,.698}
    \definecolor{linkcolor}{rgb}{.71,0.21,0.01}
    \definecolor{citecolor}{rgb}{.12,.54,.11}

    % ANSI colors
    \definecolor{ansi-black}{HTML}{3E424D}
    \definecolor{ansi-black-intense}{HTML}{282C36}
    \definecolor{ansi-red}{HTML}{E75C58}
    \definecolor{ansi-red-intense}{HTML}{B22B31}
    \definecolor{ansi-green}{HTML}{00A250}
    \definecolor{ansi-green-intense}{HTML}{007427}
    \definecolor{ansi-yellow}{HTML}{DDB62B}
    \definecolor{ansi-yellow-intense}{HTML}{B27D12}
    \definecolor{ansi-blue}{HTML}{208FFB}
    \definecolor{ansi-blue-intense}{HTML}{0065CA}
    \definecolor{ansi-magenta}{HTML}{D160C4}
    \definecolor{ansi-magenta-intense}{HTML}{A03196}
    \definecolor{ansi-cyan}{HTML}{60C6C8}
    \definecolor{ansi-cyan-intense}{HTML}{258F8F}
    \definecolor{ansi-white}{HTML}{C5C1B4}
    \definecolor{ansi-white-intense}{HTML}{A1A6B2}

    % commands and environments needed by pandoc snippets
    % extracted from the output of `pandoc -s`
    \providecommand{\tightlist}{%
      \setlength{\itemsep}{0pt}\setlength{\parskip}{0pt}}
    \DefineVerbatimEnvironment{Highlighting}{Verbatim}{commandchars=\\\{\}}
    % Add ',fontsize=\small' for more characters per line
    \newenvironment{Shaded}{}{}
    \newcommand{\KeywordTok}[1]{\textcolor[rgb]{0.00,0.44,0.13}{\textbf{{#1}}}}
    \newcommand{\DataTypeTok}[1]{\textcolor[rgb]{0.56,0.13,0.00}{{#1}}}
    \newcommand{\DecValTok}[1]{\textcolor[rgb]{0.25,0.63,0.44}{{#1}}}
    \newcommand{\BaseNTok}[1]{\textcolor[rgb]{0.25,0.63,0.44}{{#1}}}
    \newcommand{\FloatTok}[1]{\textcolor[rgb]{0.25,0.63,0.44}{{#1}}}
    \newcommand{\CharTok}[1]{\textcolor[rgb]{0.25,0.44,0.63}{{#1}}}
    \newcommand{\StringTok}[1]{\textcolor[rgb]{0.25,0.44,0.63}{{#1}}}
    \newcommand{\CommentTok}[1]{\textcolor[rgb]{0.38,0.63,0.69}{\textit{{#1}}}}
    \newcommand{\OtherTok}[1]{\textcolor[rgb]{0.00,0.44,0.13}{{#1}}}
    \newcommand{\AlertTok}[1]{\textcolor[rgb]{1.00,0.00,0.00}{\textbf{{#1}}}}
    \newcommand{\FunctionTok}[1]{\textcolor[rgb]{0.02,0.16,0.49}{{#1}}}
    \newcommand{\RegionMarkerTok}[1]{{#1}}
    \newcommand{\ErrorTok}[1]{\textcolor[rgb]{1.00,0.00,0.00}{\textbf{{#1}}}}
    \newcommand{\NormalTok}[1]{{#1}}
    
    % Additional commands for more recent versions of Pandoc
    \newcommand{\ConstantTok}[1]{\textcolor[rgb]{0.53,0.00,0.00}{{#1}}}
    \newcommand{\SpecialCharTok}[1]{\textcolor[rgb]{0.25,0.44,0.63}{{#1}}}
    \newcommand{\VerbatimStringTok}[1]{\textcolor[rgb]{0.25,0.44,0.63}{{#1}}}
    \newcommand{\SpecialStringTok}[1]{\textcolor[rgb]{0.73,0.40,0.53}{{#1}}}
    \newcommand{\ImportTok}[1]{{#1}}
    \newcommand{\DocumentationTok}[1]{\textcolor[rgb]{0.73,0.13,0.13}{\textit{{#1}}}}
    \newcommand{\AnnotationTok}[1]{\textcolor[rgb]{0.38,0.63,0.69}{\textbf{\textit{{#1}}}}}
    \newcommand{\CommentVarTok}[1]{\textcolor[rgb]{0.38,0.63,0.69}{\textbf{\textit{{#1}}}}}
    \newcommand{\VariableTok}[1]{\textcolor[rgb]{0.10,0.09,0.49}{{#1}}}
    \newcommand{\ControlFlowTok}[1]{\textcolor[rgb]{0.00,0.44,0.13}{\textbf{{#1}}}}
    \newcommand{\OperatorTok}[1]{\textcolor[rgb]{0.40,0.40,0.40}{{#1}}}
    \newcommand{\BuiltInTok}[1]{{#1}}
    \newcommand{\ExtensionTok}[1]{{#1}}
    \newcommand{\PreprocessorTok}[1]{\textcolor[rgb]{0.74,0.48,0.00}{{#1}}}
    \newcommand{\AttributeTok}[1]{\textcolor[rgb]{0.49,0.56,0.16}{{#1}}}
    \newcommand{\InformationTok}[1]{\textcolor[rgb]{0.38,0.63,0.69}{\textbf{\textit{{#1}}}}}
    \newcommand{\WarningTok}[1]{\textcolor[rgb]{0.38,0.63,0.69}{\textbf{\textit{{#1}}}}}
    
    
    % Define a nice break command that doesn't care if a line doesn't already
    % exist.
    \def\br{\hspace*{\fill} \\* }
    % Math Jax compatability definitions
    \def\gt{>}
    \def\lt{<}
    % Document parameters
    \title{pmagain}
    
    
    

    % Pygments definitions
    
\makeatletter
\def\PY@reset{\let\PY@it=\relax \let\PY@bf=\relax%
    \let\PY@ul=\relax \let\PY@tc=\relax%
    \let\PY@bc=\relax \let\PY@ff=\relax}
\def\PY@tok#1{\csname PY@tok@#1\endcsname}
\def\PY@toks#1+{\ifx\relax#1\empty\else%
    \PY@tok{#1}\expandafter\PY@toks\fi}
\def\PY@do#1{\PY@bc{\PY@tc{\PY@ul{%
    \PY@it{\PY@bf{\PY@ff{#1}}}}}}}
\def\PY#1#2{\PY@reset\PY@toks#1+\relax+\PY@do{#2}}

\expandafter\def\csname PY@tok@nn\endcsname{\let\PY@bf=\textbf\def\PY@tc##1{\textcolor[rgb]{0.00,0.00,1.00}{##1}}}
\expandafter\def\csname PY@tok@m\endcsname{\def\PY@tc##1{\textcolor[rgb]{0.40,0.40,0.40}{##1}}}
\expandafter\def\csname PY@tok@cp\endcsname{\def\PY@tc##1{\textcolor[rgb]{0.74,0.48,0.00}{##1}}}
\expandafter\def\csname PY@tok@c\endcsname{\let\PY@it=\textit\def\PY@tc##1{\textcolor[rgb]{0.25,0.50,0.50}{##1}}}
\expandafter\def\csname PY@tok@s2\endcsname{\def\PY@tc##1{\textcolor[rgb]{0.73,0.13,0.13}{##1}}}
\expandafter\def\csname PY@tok@sh\endcsname{\def\PY@tc##1{\textcolor[rgb]{0.73,0.13,0.13}{##1}}}
\expandafter\def\csname PY@tok@nv\endcsname{\def\PY@tc##1{\textcolor[rgb]{0.10,0.09,0.49}{##1}}}
\expandafter\def\csname PY@tok@se\endcsname{\let\PY@bf=\textbf\def\PY@tc##1{\textcolor[rgb]{0.73,0.40,0.13}{##1}}}
\expandafter\def\csname PY@tok@cpf\endcsname{\let\PY@it=\textit\def\PY@tc##1{\textcolor[rgb]{0.25,0.50,0.50}{##1}}}
\expandafter\def\csname PY@tok@mi\endcsname{\def\PY@tc##1{\textcolor[rgb]{0.40,0.40,0.40}{##1}}}
\expandafter\def\csname PY@tok@nf\endcsname{\def\PY@tc##1{\textcolor[rgb]{0.00,0.00,1.00}{##1}}}
\expandafter\def\csname PY@tok@mh\endcsname{\def\PY@tc##1{\textcolor[rgb]{0.40,0.40,0.40}{##1}}}
\expandafter\def\csname PY@tok@kc\endcsname{\let\PY@bf=\textbf\def\PY@tc##1{\textcolor[rgb]{0.00,0.50,0.00}{##1}}}
\expandafter\def\csname PY@tok@gt\endcsname{\def\PY@tc##1{\textcolor[rgb]{0.00,0.27,0.87}{##1}}}
\expandafter\def\csname PY@tok@o\endcsname{\def\PY@tc##1{\textcolor[rgb]{0.40,0.40,0.40}{##1}}}
\expandafter\def\csname PY@tok@nl\endcsname{\def\PY@tc##1{\textcolor[rgb]{0.63,0.63,0.00}{##1}}}
\expandafter\def\csname PY@tok@kr\endcsname{\let\PY@bf=\textbf\def\PY@tc##1{\textcolor[rgb]{0.00,0.50,0.00}{##1}}}
\expandafter\def\csname PY@tok@nb\endcsname{\def\PY@tc##1{\textcolor[rgb]{0.00,0.50,0.00}{##1}}}
\expandafter\def\csname PY@tok@gr\endcsname{\def\PY@tc##1{\textcolor[rgb]{1.00,0.00,0.00}{##1}}}
\expandafter\def\csname PY@tok@sr\endcsname{\def\PY@tc##1{\textcolor[rgb]{0.73,0.40,0.53}{##1}}}
\expandafter\def\csname PY@tok@vg\endcsname{\def\PY@tc##1{\textcolor[rgb]{0.10,0.09,0.49}{##1}}}
\expandafter\def\csname PY@tok@c1\endcsname{\let\PY@it=\textit\def\PY@tc##1{\textcolor[rgb]{0.25,0.50,0.50}{##1}}}
\expandafter\def\csname PY@tok@il\endcsname{\def\PY@tc##1{\textcolor[rgb]{0.40,0.40,0.40}{##1}}}
\expandafter\def\csname PY@tok@gd\endcsname{\def\PY@tc##1{\textcolor[rgb]{0.63,0.00,0.00}{##1}}}
\expandafter\def\csname PY@tok@mf\endcsname{\def\PY@tc##1{\textcolor[rgb]{0.40,0.40,0.40}{##1}}}
\expandafter\def\csname PY@tok@s\endcsname{\def\PY@tc##1{\textcolor[rgb]{0.73,0.13,0.13}{##1}}}
\expandafter\def\csname PY@tok@gh\endcsname{\let\PY@bf=\textbf\def\PY@tc##1{\textcolor[rgb]{0.00,0.00,0.50}{##1}}}
\expandafter\def\csname PY@tok@na\endcsname{\def\PY@tc##1{\textcolor[rgb]{0.49,0.56,0.16}{##1}}}
\expandafter\def\csname PY@tok@cm\endcsname{\let\PY@it=\textit\def\PY@tc##1{\textcolor[rgb]{0.25,0.50,0.50}{##1}}}
\expandafter\def\csname PY@tok@vc\endcsname{\def\PY@tc##1{\textcolor[rgb]{0.10,0.09,0.49}{##1}}}
\expandafter\def\csname PY@tok@kd\endcsname{\let\PY@bf=\textbf\def\PY@tc##1{\textcolor[rgb]{0.00,0.50,0.00}{##1}}}
\expandafter\def\csname PY@tok@nc\endcsname{\let\PY@bf=\textbf\def\PY@tc##1{\textcolor[rgb]{0.00,0.00,1.00}{##1}}}
\expandafter\def\csname PY@tok@si\endcsname{\let\PY@bf=\textbf\def\PY@tc##1{\textcolor[rgb]{0.73,0.40,0.53}{##1}}}
\expandafter\def\csname PY@tok@mb\endcsname{\def\PY@tc##1{\textcolor[rgb]{0.40,0.40,0.40}{##1}}}
\expandafter\def\csname PY@tok@ch\endcsname{\let\PY@it=\textit\def\PY@tc##1{\textcolor[rgb]{0.25,0.50,0.50}{##1}}}
\expandafter\def\csname PY@tok@s1\endcsname{\def\PY@tc##1{\textcolor[rgb]{0.73,0.13,0.13}{##1}}}
\expandafter\def\csname PY@tok@sb\endcsname{\def\PY@tc##1{\textcolor[rgb]{0.73,0.13,0.13}{##1}}}
\expandafter\def\csname PY@tok@gu\endcsname{\let\PY@bf=\textbf\def\PY@tc##1{\textcolor[rgb]{0.50,0.00,0.50}{##1}}}
\expandafter\def\csname PY@tok@bp\endcsname{\def\PY@tc##1{\textcolor[rgb]{0.00,0.50,0.00}{##1}}}
\expandafter\def\csname PY@tok@mo\endcsname{\def\PY@tc##1{\textcolor[rgb]{0.40,0.40,0.40}{##1}}}
\expandafter\def\csname PY@tok@gs\endcsname{\let\PY@bf=\textbf}
\expandafter\def\csname PY@tok@kp\endcsname{\def\PY@tc##1{\textcolor[rgb]{0.00,0.50,0.00}{##1}}}
\expandafter\def\csname PY@tok@ne\endcsname{\let\PY@bf=\textbf\def\PY@tc##1{\textcolor[rgb]{0.82,0.25,0.23}{##1}}}
\expandafter\def\csname PY@tok@w\endcsname{\def\PY@tc##1{\textcolor[rgb]{0.73,0.73,0.73}{##1}}}
\expandafter\def\csname PY@tok@gp\endcsname{\let\PY@bf=\textbf\def\PY@tc##1{\textcolor[rgb]{0.00,0.00,0.50}{##1}}}
\expandafter\def\csname PY@tok@go\endcsname{\def\PY@tc##1{\textcolor[rgb]{0.53,0.53,0.53}{##1}}}
\expandafter\def\csname PY@tok@ni\endcsname{\let\PY@bf=\textbf\def\PY@tc##1{\textcolor[rgb]{0.60,0.60,0.60}{##1}}}
\expandafter\def\csname PY@tok@k\endcsname{\let\PY@bf=\textbf\def\PY@tc##1{\textcolor[rgb]{0.00,0.50,0.00}{##1}}}
\expandafter\def\csname PY@tok@vi\endcsname{\def\PY@tc##1{\textcolor[rgb]{0.10,0.09,0.49}{##1}}}
\expandafter\def\csname PY@tok@err\endcsname{\def\PY@bc##1{\setlength{\fboxsep}{0pt}\fcolorbox[rgb]{1.00,0.00,0.00}{1,1,1}{\strut ##1}}}
\expandafter\def\csname PY@tok@ow\endcsname{\let\PY@bf=\textbf\def\PY@tc##1{\textcolor[rgb]{0.67,0.13,1.00}{##1}}}
\expandafter\def\csname PY@tok@sd\endcsname{\let\PY@it=\textit\def\PY@tc##1{\textcolor[rgb]{0.73,0.13,0.13}{##1}}}
\expandafter\def\csname PY@tok@no\endcsname{\def\PY@tc##1{\textcolor[rgb]{0.53,0.00,0.00}{##1}}}
\expandafter\def\csname PY@tok@cs\endcsname{\let\PY@it=\textit\def\PY@tc##1{\textcolor[rgb]{0.25,0.50,0.50}{##1}}}
\expandafter\def\csname PY@tok@gi\endcsname{\def\PY@tc##1{\textcolor[rgb]{0.00,0.63,0.00}{##1}}}
\expandafter\def\csname PY@tok@nd\endcsname{\def\PY@tc##1{\textcolor[rgb]{0.67,0.13,1.00}{##1}}}
\expandafter\def\csname PY@tok@sx\endcsname{\def\PY@tc##1{\textcolor[rgb]{0.00,0.50,0.00}{##1}}}
\expandafter\def\csname PY@tok@nt\endcsname{\let\PY@bf=\textbf\def\PY@tc##1{\textcolor[rgb]{0.00,0.50,0.00}{##1}}}
\expandafter\def\csname PY@tok@kt\endcsname{\def\PY@tc##1{\textcolor[rgb]{0.69,0.00,0.25}{##1}}}
\expandafter\def\csname PY@tok@kn\endcsname{\let\PY@bf=\textbf\def\PY@tc##1{\textcolor[rgb]{0.00,0.50,0.00}{##1}}}
\expandafter\def\csname PY@tok@sc\endcsname{\def\PY@tc##1{\textcolor[rgb]{0.73,0.13,0.13}{##1}}}
\expandafter\def\csname PY@tok@ge\endcsname{\let\PY@it=\textit}
\expandafter\def\csname PY@tok@ss\endcsname{\def\PY@tc##1{\textcolor[rgb]{0.10,0.09,0.49}{##1}}}

\def\PYZbs{\char`\\}
\def\PYZus{\char`\_}
\def\PYZob{\char`\{}
\def\PYZcb{\char`\}}
\def\PYZca{\char`\^}
\def\PYZam{\char`\&}
\def\PYZlt{\char`\<}
\def\PYZgt{\char`\>}
\def\PYZsh{\char`\#}
\def\PYZpc{\char`\%}
\def\PYZdl{\char`\$}
\def\PYZhy{\char`\-}
\def\PYZsq{\char`\'}
\def\PYZdq{\char`\"}
\def\PYZti{\char`\~}
% for compatibility with earlier versions
\def\PYZat{@}
\def\PYZlb{[}
\def\PYZrb{]}
\makeatother


    % Exact colors from NB
    \definecolor{incolor}{rgb}{0.0, 0.0, 0.5}
    \definecolor{outcolor}{rgb}{0.545, 0.0, 0.0}



    
    % Prevent overflowing lines due to hard-to-break entities
    \sloppy 
    % Setup hyperref package
    \hypersetup{
      breaklinks=true,  % so long urls are correctly broken across lines
      colorlinks=true,
      urlcolor=urlcolor,
      linkcolor=linkcolor,
      citecolor=citecolor,
      }
    % Slightly bigger margins than the latex defaults
    
    \geometry{verbose,tmargin=1in,bmargin=1in,lmargin=1in,rmargin=1in}
    
    

    \begin{document}
    
    
    \author{Patrick Allo}\affiliation{University of Oxford}\title{Polymath Again and Differently}

\date{\today}
\maketitle

    
    

    







    \begin{Verbatim}[commandchars=\\\{\}]
starting at:  2016-09-09 22:37:53.597574
Polymath 1 processed in  0:01:32.127023
Polymath 2 processed in  0:00:01.533501

    \end{Verbatim}

    \begin{Verbatim}[commandchars=\\\{\}]
WARNING:root:Skipped clustering for The Polynomial Hirsch Conjecture, a Proposal for Polymath3 (Cont.), only 2 comments
WARNING:root:Skipped clustering for Plans for polymath3, only 2 comments
WARNING:root:Skipped clustering for The Polynomial Hirsch Conjecture: The Crux of the Matter., only 1 comments

    \end{Verbatim}

    \begin{Verbatim}[commandchars=\\\{\}]
Polymath 3 processed in  0:00:31.168524
Polymath 4 processed in  0:00:20.902427

    \end{Verbatim}

    \begin{Verbatim}[commandchars=\\\{\}]
WARNING:root:Skipped clustering for EDP19 — removing some vagueness, only 4 comments
WARNING:root:Skipped clustering for EDP: a possible revival, only 2 comments
WARNING:root:Node comment-30735 not in network

    \end{Verbatim}

    \begin{Verbatim}[commandchars=\\\{\}]
Polymath 5 processed in  0:03:50.910960
Polymath 6 processed in  0:00:06.358388

    \end{Verbatim}

    \begin{Verbatim}[commandchars=\\\{\}]
WARNING:root:Skipped clustering for Polymath7 research thread 5: the hot spots conjecture, only 5 comments

    \end{Verbatim}

    \begin{Verbatim}[commandchars=\\\{\}]
Polymath 7 processed in  0:00:30.067585

    \end{Verbatim}

    \begin{Verbatim}[commandchars=\\\{\}]
WARNING:root:Node comment-237793 not in network
WARNING:root:Node comment-239914 not in network
WARNING:root:Node comment-238422 not in network
WARNING:root:Node comment-285679 not in network
WARNING:root:Node comment-270633 not in network
WARNING:root:Node comment-297930 not in network
WARNING:bs4.dammit:Some characters could not be decoded, and were replaced with REPLACEMENT CHARACTER.
WARNING:root:Node comment-428034 not in network

    \end{Verbatim}

    \begin{Verbatim}[commandchars=\\\{\}]
Polymath 8 processed in  0:06:33.412183
Polymath 9 processed in  0:00:18.361485

    \end{Verbatim}

    \begin{Verbatim}[commandchars=\\\{\}]
WARNING:root:Node comment-24559 not in network
WARNING:root:Moving on without removing comments for https://gilkalai.wordpress.com/2016/05/17/polymath-10-emergency-post-5-the-erdos-szemeredi-sunflower-conjecture-is-now-proven/
WARNING:root:Moving on without removing comments for https://gilkalai.wordpress.com/2016/05/27/polymath-10-post-6-the-erdos-rado-sunflower-conjecture-and-the-turan-43-problem-homological-approaches/
WARNING:root:Skipped clustering for Polymath 10 post 6: The Erdos-Rado sunflower conjecture,  and the Turan (4,3) problem: homological approaches., only 4 comments
WARNING:root:Threads not in proper order

    \end{Verbatim}

    \begin{Verbatim}[commandchars=\\\{\}]
Polymath 10 processed in  0:00:37.843350

    \end{Verbatim}

    \begin{Verbatim}[commandchars=\\\{\}]
WARNING:root:Node comment-156569 not in network
WARNING:root:Threads not in proper order

    \end{Verbatim}

    \begin{Verbatim}[commandchars=\\\{\}]
Polymath 11 processed in  0:00:38.435201
completed at:  2016-09-09 22:52:54.719929

    \end{Verbatim}













    \textbf{To Do: add feature to get first and last from info in PM\_FRAME}

    \section{Introduction}\label{introduction}

Within the philosophy of mathematical practice the Polymath projects,
the online collaborative projects initiated by Timothy Gowers, have
since their start in 2009 attracted visible attention
\cite{763884/WXE3J6GJ}, \cite{763884/RXTHUZGM}, \cite{763884/ICNJ8I5P}.
As the philosophy of mathematical practice is concerned with how
mathematics is done, evaluated, and applied, this interest should barely
surprise us. The research-activity within Polymath was carried out
publicly on several blogs, and thus led to the creation of a large
repository of interactive mathematical practice in action: a treasure of
information ready to be explored. In addition, the main players in this
project (the Field-medallists Timothy Gowers and Terrence Tao)
continuously reflected on the enterprise, and provided additional
insight on the nature of mathematical research, and large-scale
collaboration \cite{763884/QSM6KVB6} §2. This led amongst others to the
claim that the online collective problem solving that underpins the
Polymath-projects consists of ``mathematical research in a new way''
\cite{763884/TVHHE2TD}: 879.

The upshot of our inquiry is to improve on prior in-depth examinations
of the Polymath projects \cite{763884/BQAZU5GI}, \cite{763884/9KE9SA9C},
\cite{763884/BBVPZ6NT}, \cite{763884/HM3CT69N}, \cite{763884/RXTHUZGM}
by taking advantage of the fact that we can now look back on 11
Polymath-projects (apart from \cite{Ball2014} and Franzonia \& Sauermann
2014, most analysis and discussion focused either on the first project,
or on one of the smaller \emph{mini} projects), two of which only
started within the last year, and by relying on scholarship on
e-research and citizen science that isn't typically part of the
theoretical background that the philosophy of mathematical practice
builds on. Specifically, we wish to:

\begin{enumerate}
\def\labelenumi{\arabic{enumi}.}
\item
  Rely on more data, and hence look beyond the intial succes and broad
  appeal of Polymath 1 and try to indentify long-term features of
  Polymath. Such features may concern the \emph{ad hoc} community
  created by the Polymath projects, but also the patterns of
  collaboration that are arise as ``weblog mathematics'' becomes more
  mature, or the question of what Polymath is crowdsourcing (pieces of
  mathematical proofs, ideas and insights, or the community itself).
\item
  Put Polymath in context by building on the social study of e-research,
  crowd-science and citizen science, and answer questions regarding the
  scale of the Polymath projects, the role of technology, and the
  division of labour between the organisers and the participants to the
  Polymath projects by contrasting our data and insights with available
  data and insights on other online collaborative scientific enterprises
  that have been studied.
\item
  Engage with the literature on the structure of scientific communities
  that has been developed within (formally oriented) social epistemology
  and the philosophy of science, and contribute to the current debate on
  how we should provide an empirical grounding for the \emph{a priori}
  methods (e.g.~simulations based on simplified models of scientific
  communities).
\end{enumerate}

With most weight placed on the first two items, and the third item
relegated to a dedicated study based on the same empirical material.

    Notwithstanding our reliance on online research methods, our final
intent is still to contribute to the philosophy of mathematical
practices by adopting a sociological approach and relying on empirical
data. This approach is not without risks.

First, we still need to ensure that the perspective we adopt allows us
to answer the questions \textbf{to be completed}

\begin{itemize}
\tightlist
\item
  Kuhn's point (used by Gilies) that if there's a community, there's a
  sociology. This explains why sociological methods can be used to look
  at Polymath, but not yet how this can contribute to the philosophy of
  mathematics.
\end{itemize}

Second, there is a kind of paradox \textbf{to be completed}

    \subsection{Overview of the paper}\label{overview-of-the-paper}

\begin{enumerate}
\def\labelenumi{\arabic{enumi}.}
\tightlist
\item
  Polymath: an informal overview
\item
  Polymath in perspective

  \begin{enumerate}
  \def\labelenumii{\arabic{enumii}.}
  \tightlist
  \item
    e-research, crowd-science and citizen science
  \item
    The role of ICT's
  \end{enumerate}
\item
  A network-analysis of the Polymath-projects
\item
  Polymath as a scientific community
\item
  Investigating individual projects

  \begin{enumerate}
  \def\labelenumii{\arabic{enumii}.}
  \tightlist
  \item
    Polymath 1
  \item
    One of the standard projects (3,4 or 5)
  \item
    Polymath 8
  \end{enumerate}
\end{enumerate}

    \section{Polymath: an informal
overview}\label{polymath-an-informal-overview}

The story of how Polymath started has been retold in many places,
including reports by the leading figures \cite{Gowers:Nature:2009},
\cite{Gowers:AnIrregularMind:2010}, and
\cite{Nielsen:ReinventingDiscoveryTheNewEraOf:2012}, in the popular
science press (New Scientist: ``Mathematics becomes more sociable'',
``How to build the global mathematics brain''.), and within the
philosophy of mathematical practices
\cite{VanBendegem:MathematicsAndTheNewTechnologiesPartIii:}.

We do not need to repeat this story in full detail, but only quote
briefly from Gower's
\href{https://gowers.wordpress.com/2009/01/27/is-massively-collaborative-mathematics-possible/}{``Is
massively collaborative mathematics possible?''} blog-post that launched
the idea and described the intended model of collaboration:

\begin{quote}
It seems to me that, at least in theory, a different model could work:
different, that is, from the usual model of people working in isolation
or collaborating with one or two others. Suppose one had a forum (in the
non-technical sense, but quite possibly in the technical sense as well)
for the online discussion of a particular problem. The idea would be
that anybody who had anything whatsoever to say about the problem could
chip in. And the ethos of the forum --- in whatever form it took ---
would be that comments would mostly be kept short. In other words, what
you would not tend to do, at least if you wanted to keep within the
spirit of things, is spend a month thinking hard about the problem and
then come back and write ten pages about it. Rather, you would
contribute ideas even if they were undeveloped and/or likely to be
wrong.
\end{quote}

and also suggested why \emph{blog-mathematics} might be a good idea:

\begin{quote}
Here is where the beauty of blogs, wikis, forums etc. comes in: they are
completely public, as is their entire history. To see what effect this
might have, imagine that a problem was being solved via comments on a
blog post. Suppose that the blog was pretty active and that the post was
getting several interesting comments. And suppose that you had an idea
that you thought might be a good one. Instead of the usual reaction of
being afraid to share it in case someone else beat you to the solution,
you would be afraid not to share it in case someone beat you to that
particular idea. And if the problem eventually got solved, and published
under some pseudonym like Polymath, say, with a footnote linking to the
blog and explaining how the problem had been solved, then anybody could
go to the blog and look at all the comments. And there they would find
your idea and would know precisely what you had contributed. There might
be arguments about which ideas had proved to be most important to the
solution, but at least all the evidence would be there for everybody to
look at.
\end{quote}

    Prior scholarship devoted to the Polymath-projects can be organised in
three disciplinary categories. A first category is dedicated toi how
specific information-technologies are used for collaboration, and can
naturally by related to hypertext research (in particular the use of
blogs and wiki's) and humanistic computing. This category includes the
above-mentioned work of Barany, Cranshaw \& Kittur, and Varshney. The
authors of these papers try to identify specific features of how
collaboration in the Polymath-projects proceeds (e.g.~remarks on
noise-to-signal ratio by Barany, or the importance of leadership by
Barany and Cranshaw \& Kittur), propose criteria for success, and seek
to evaluate to what extent collaboration in Polymath is successful.
Unlike the present paper, the focus in this work is limited to the first
Polymath-project. By contrast, it relies on social science methodology,
and network-analysis in particular, as will be done in the present
paper.

A second category squarely belongs to the philosophy of mathematical
practice. This includes work by Martin and Pease, with its emphasis on
problem-solving and argumentation in mathematics (influenced by Lakatos,
Pólya, and by work in formal argumentation), and focus on the use of
ICT's in mathematics that relates to the metaphor of \emph{social
machines}. Specifically, they analysed the third mini-Polymath project
(see also below) and identified categories of comments and the
mathematical concepts they relied on, and developed a typology of
questions based on a sample of questions asked on MathOverflow (the
research-mathematics site of StackExchange). Previous work by the
present authors also belongs to this category. Finally, though not
explicitly part of the philosophy of mathematical practice,
\textbf{Patterson et al.} have also drawn attention to the value of
Polymath as a record that publicly exposes the intermediate stages of
research and hence reveals how mathematical inquiry (and collaboration)
proceeds, and to the possibility of applying network-science to study
the development of these projects.

In the third category we find insider and popular science press reports
on Polymath, which were the decisive element in how Polymath gained
notoriety, and include in the first place Gowers and Nielsen's opinion
piece in Nature \cite{Gowers:Nature:2009} and Nielsen's ``Reinventing
Discovery: the New Era of Networked Science'' in which Polymath featured
prominently \cite{Nielsen:ReinventingDiscoveryTheNewEraOf:2012}.
Additional inside-reports include \cite{Nielsen2010} (though Nielsen
describes himself as an ``interested outsider'', p.~651 \emph{op. cit})
and \cite{Gowers:AnIrregularMind:2010} which covered the problem solved
in the first Polymath-project, the views about collaboration in
mathematics that motivated the project, and lessons learned from
organising and guiding a large collaborative project. Finally, articles
in
\href{https://www.newscientist.com/article/mg21028113-900-how-to-build-the-global-mathematics-brain/}{New
Scientist} (\cite{aron2011build}), Scientific American
(\cite{castelvecchi2010problem}), The Mathematical Intelligencer
(\cite{Nathanson}), a more recent short piece in Nature by Philip Ball
(\cite{Ball2014}), and many blog-posts complete this category.

In addition to these three categories, the Polymath-projects are also
addressed in the more general literature on e-research and
crowd-science. Section 3 is entirely devoted to this aspect of Polymath.

    Since its start in early 2009, 11 full Polymath projects were started,
and 4 smaller, so-called Mini-Polymath, projects were started. We focus
here on the former group, dedicated to research-problems, and leave the
latter (based on questions from the Mathematics Olympiad Finals) aside
\textbf{(but see Martin \& Pease, and Patterson et al.)}. Of these
projects, 3 were unambiguously succesfull (1, 4, and 8) and led to
published results, whereas another one led to insights that were used in
published work (5). Whereas the two most recent projects can still be
considered active, and Polymath 2 (failed attempts to revive) and
Polymath 9 (which turned out to be based on mistaken assumptions) can be
considered as closed, the status of remaining projects is harder to
determine.

Already during the first project the discussion was not limited to a
single blog, and Polymath-projects have since been conducted on 5
different blogs. A general overview of the projects, as well as
summaries of background-material and progress is collected in a
dedicated
\href{http://www.michaelnielsen.org/polymath1/index.php?title=Main_Page}{Wiki}
hosted by Michael Nielsen. The main Polymath-blog also serves as a hub
to announce new initiatives, and crowdsource ideas for future projects.
Such hubs are no doubt essential given how Polymath-efforts and
initiatives are distributed, and given their openness to new
initiatives. In September 2015 Gil Kalai also created a
\href{http://mathoverflow.net/questions/219638/proposals-for-polymath-projects}{question}
on MathOverflow (the StackExchange community for professional
mathematicians) devoted to new Polymath-proposals. This serves again as
a new hub, and takes advantage of the more elaborate platform,
especially the ability to vote proposals up or down, offered by
StackExchange.

A brief overview of the projects is included below:

\begin{longtable}[c]{@{}llll@{}}
\toprule
\begin{minipage}[b]{0.11\columnwidth}\raggedright\strut
Project
\strut\end{minipage} &
\begin{minipage}[b]{0.10\columnwidth}\raggedright\strut
Title
\strut\end{minipage} &
\begin{minipage}[b]{0.09\columnwidth}\raggedright\strut
~Start
\strut\end{minipage} &
\begin{minipage}[b]{0.09\columnwidth}\raggedright\strut
Status
\strut\end{minipage}\tabularnewline
\midrule
\endhead
\begin{minipage}[t]{0.11\columnwidth}\raggedright\strut
Polymath1
\strut\end{minipage} &
\begin{minipage}[t]{0.10\columnwidth}\raggedright\strut
``New proofs and bounds for the density Hales-Jewett theorem.''
\strut\end{minipage} &
\begin{minipage}[t]{0.09\columnwidth}\raggedright\strut
Initiated Feb 1, 2009
\strut\end{minipage} &
\begin{minipage}[t]{0.09\columnwidth}\raggedright\strut
research results have now been published.
\strut\end{minipage}\tabularnewline
\begin{minipage}[t]{0.11\columnwidth}\raggedright\strut
Polymath2
\strut\end{minipage} &
\begin{minipage}[t]{0.10\columnwidth}\raggedright\strut
``Must an ``explicitly defined'' Banach space contain c\_0 or l\_p?''
\strut\end{minipage} &
\begin{minipage}[t]{0.09\columnwidth}\raggedright\strut
Initiated Feb 17, 2009
\strut\end{minipage} &
\begin{minipage}[t]{0.09\columnwidth}\raggedright\strut
attempts to relaunch via wiki, June 9 2010.
\strut\end{minipage}\tabularnewline
\begin{minipage}[t]{0.11\columnwidth}\raggedright\strut
Polymath3
\strut\end{minipage} &
\begin{minipage}[t]{0.10\columnwidth}\raggedright\strut
``The polynomial Hirsch conjecture.''
\strut\end{minipage} &
\begin{minipage}[t]{0.09\columnwidth}\raggedright\strut
Proposed July 17, 2009; launched, September 30, 2010.
\strut\end{minipage} &
\begin{minipage}[t]{0.09\columnwidth}\raggedright\strut
Undecided
\strut\end{minipage}\tabularnewline
\begin{minipage}[t]{0.11\columnwidth}\raggedright\strut
Polymath4
\strut\end{minipage} &
\begin{minipage}[t]{0.10\columnwidth}\raggedright\strut
``A deterministic way to find primes.''
\strut\end{minipage} &
\begin{minipage}[t]{0.09\columnwidth}\raggedright\strut
Proposed July 27, 2009; launched Aug 9, 2009
\strut\end{minipage} &
\begin{minipage}[t]{0.09\columnwidth}\raggedright\strut
Research results have now been published.
\strut\end{minipage}\tabularnewline
\begin{minipage}[t]{0.11\columnwidth}\raggedright\strut
Polymath5
\strut\end{minipage} &
\begin{minipage}[t]{0.10\columnwidth}\raggedright\strut
``The Erdős discrepancy problem.''
\strut\end{minipage} &
\begin{minipage}[t]{0.09\columnwidth}\raggedright\strut
Proposed Jan 10, 2010; launched Jan 19, 2010
\strut\end{minipage} &
\begin{minipage}[t]{0.09\columnwidth}\raggedright\strut
Activity ceased by the end of 2012, but results from the project were
used to solve the problem in 2015.
\strut\end{minipage}\tabularnewline
\begin{minipage}[t]{0.11\columnwidth}\raggedright\strut
Polymath6
\strut\end{minipage} &
\begin{minipage}[t]{0.10\columnwidth}\raggedright\strut
``Improving the bounds for Roth's theorem.''
\strut\end{minipage} &
\begin{minipage}[t]{0.09\columnwidth}\raggedright\strut
Proposed Feb 5, 2011
\strut\end{minipage} &
\begin{minipage}[t]{0.09\columnwidth}\raggedright\strut
Partial result published by non-participant
\strut\end{minipage}\tabularnewline
\begin{minipage}[t]{0.11\columnwidth}\raggedright\strut
Polymath7
\strut\end{minipage} &
\begin{minipage}[t]{0.10\columnwidth}\raggedright\strut
``Establishing the Hot Spots conjecture for acute-angled triangles.''
\strut\end{minipage} &
\begin{minipage}[t]{0.09\columnwidth}\raggedright\strut
Proposed, May 31st, 2012; launched, Jun 8, 2012.
\strut\end{minipage} &
\begin{minipage}[t]{0.09\columnwidth}\raggedright\strut
Undecided
\strut\end{minipage}\tabularnewline
\begin{minipage}[t]{0.11\columnwidth}\raggedright\strut
Polymath8
\strut\end{minipage} &
\begin{minipage}[t]{0.10\columnwidth}\raggedright\strut
``Improving the bounds for small gaps between primes.''
\strut\end{minipage} &
\begin{minipage}[t]{0.09\columnwidth}\raggedright\strut
Proposed, June 4, 2013; launched, June 4, 2013.
\strut\end{minipage} &
\begin{minipage}[t]{0.09\columnwidth}\raggedright\strut
Research results have now been published.
\strut\end{minipage}\tabularnewline
\begin{minipage}[t]{0.11\columnwidth}\raggedright\strut
Polymath9
\strut\end{minipage} &
\begin{minipage}[t]{0.10\columnwidth}\raggedright\strut
``Exploring Borel determinacy-based methods for giving complexity
bounds.''
\strut\end{minipage} &
\begin{minipage}[t]{0.09\columnwidth}\raggedright\strut
Proposed, Oct 24, 2013; launched, Nov 3, 2013
\strut\end{minipage} &
\begin{minipage}[t]{0.09\columnwidth}\raggedright\strut
``success of a kind''.
\strut\end{minipage}\tabularnewline
\begin{minipage}[t]{0.11\columnwidth}\raggedright\strut
Polymath10
\strut\end{minipage} &
\begin{minipage}[t]{0.10\columnwidth}\raggedright\strut
``Improving the bounds for the Erdos-Rado sunflower lemma.''
\strut\end{minipage} &
\begin{minipage}[t]{0.09\columnwidth}\raggedright\strut
Launched, Nov 2, 2015
\strut\end{minipage} &
\begin{minipage}[t]{0.09\columnwidth}\raggedright\strut
ongoing
\strut\end{minipage}\tabularnewline
\begin{minipage}[t]{0.11\columnwidth}\raggedright\strut
Polymath11
\strut\end{minipage} &
\begin{minipage}[t]{0.10\columnwidth}\raggedright\strut
``Proving Frankl's union-closed conjecture.''
\strut\end{minipage} &
\begin{minipage}[t]{0.09\columnwidth}\raggedright\strut
Proposed Jan 21, 2016; launched Jan 29, 2016
\strut\end{minipage} &
\begin{minipage}[t]{0.09\columnwidth}\raggedright\strut
ongoing
\strut\end{minipage}\tabularnewline
\bottomrule
\end{longtable}

The reach and size of each of these projects varied considerably. The
plots below summarise the participation-data and compare the different
projects by, first, looking at the number of participants and comments
in each project, and, secondly, looking at how active participants were
in each project.

Many of the projects are organised in separate research-threads limited
to discussing how a given mathematical problem could be solved and
discussion-threads devoted to meta-aspects and reflections on both the
problem and the process of collaboration. As we will see, both types of
threads proceeded differently, and it is often instructive to consider
the relevant data separately.



    \begin{figure*}
        \begin{center}\adjustimage{max size={0.9\linewidth}{0.4\paperheight}}{pmagain_files/pmagain_27_0.png}\end{center}
        \caption{Size of different Polymath Projects}
        \label{fig:overview}
    \end{figure*}
    

    \begin{figure}
        \begin{center}\adjustimage{max size={0.9\linewidth}{0.4\paperheight}}{pmagain_files/pmagain_28_0.png}\end{center}
        \caption{}
        \label{}
    \end{figure}
    
    Based on the total number of participants (the first figure), the first
Polymath-project remains exceptional as the project with the second
highest number of participants (only Polymath 8, an outlier in many
respects, attracted more participants), and, especially, as the project
with the broadest appeal. More than half of the Polymath 1 participants
were only active in the discussion-threads, and hence presumably
constituted a following that exceeded the intended audience of
mathematicians with the required expertise. If we only look at
participants to research-threads, we see that most projects that
unambiguously \emph{took of} (i.e.~generated more than 100 comments)
attracted approximately between 25 and 56 participants, whereas Polymath
8 attracted almost 3 times as many participants (149).

The box-plots in the second figure reveals a totally different aspect of
Polymath-participation, namely that (independently of the types of
threads we consider) the average commenting-activity invariable stays
below 10, whereas the highly active participants are systematically
outliers (note the use of a logarithmic scale for the y-axis). This
suggests that even when projects attract a large number of participants,
a very small core is responsible for a large proportion of the comments.
In the case of Polymath 8, with more than 100 participants in the
research-threads, there are 20 extreme outliers who contributed 88\% of
all the comments (2396 from a total of 2717). This presence of extreme
outliers is discernible in all projects that generated more than 100
comments.

    \section{Polymath in perspective}\label{polymath-in-perspective}

\subsection{e-research, crowd-science and citizen
science}\label{e-research-crowd-science-and-citizen-science}

The Polymath-projects clearly fit the definition of e-research as
research based on the use of digital tools and data---and the internet
in particular---for the distributed and collaborative production of
knowledge (Meyer and Schroeder 2015 chapt. 1). Because it makes
project-participation and access to intermediate inputs entirely open,
it also qualifies as a type of crowd-science as conceived in Franzoni \&
Sauermann 2014 (see especially the taxonomy of knowledge production
regimes Fig. 3). Finally, because it is in principle open to amateur
mathematicians, it is also a kind of citizen-science. At the same time,
Polymath remains in several respects an outlier, which presumably
accounts for its relative absence from the literature in this field
(except for Nielsen's ``Reinventing Discovery'' and Franzoni \&
Sauermann 2014). We review some of the reasons why Polymath is
exceptional to situate it within the broader landscape:

First, unless we compare Polymath to paper and pencil mathematics, the
Polymath-projects are by contemporary standards a fairly low-tech
enterprise. It relies on off-the-shelf tools, namely freely available
blogging-platforms (and to a lesser extent Wiki software), and became
possible because of the availability of a LaTeX-plugin for WordPress
(\cite{Gowers:AnIrregularMind:2010},
\href{https://en.blog.wordpress.com/2007/02/17/math-for-the-masses/}{announcement
by Mike Adams},
\href{https://terrytao.wordpress.com/2009/07/22/imo-2009-q6-mini-polymath-project-impressions-reflections-analysis/}{reference
to the plugin by Terry
Tao},\href{https://blog.jonudell.net/2009/07/31/polymath-equals-user-innovatio/}{general
discussion by Jon Udell}).

This is different from most other types of e-research which typically
rely ion the development of dedicated tools for collaboration (see the
examples in \cite{Nielsen:ReinventingDiscoveryTheNewEraOf:2012} and
\cite{meyer2015knowledge}), and have in the case of Zooniverse even led
to the creation of a re-usable tool or platform for online collaboration
(https://www.zooniverse.org/about). Polymath, by contrast, has become a
re-usable idea that is primarily defined by the rules Gowers initially
formulated (a recipe , and is only to a lesser extent characterised by
the specific technology it relies on (blogs, but also wikis). It's
appeal also remains limited to the formal sciences, with proposals to
implement the kind of open collaboration Polymath initiated in AI
(\textbf{Zadrozny, de Paiva and Moss 2015}) and as a means to create
research-opportunities for mathematics undergraduates (\textbf{Parsley
and Rusinko 2016}).

Second, Polymath is clearly a small-scale enterprise when compared to
many collaborative online projects. The largest Polymath-project
(Polymath 8) attracted 180 participants, whereas the smaller projects (2
and 6) did not even reach 20 participants. Over the 11 projects 465
distinct participants were identified (excluding ``anonymous'' and its
variants). By comparison, the 7 Zooniverse-projects reviewed in
\cite{Sauermann Ranzoni 2015} reached between 3,186 and 28,828
participants in their first 180 days. This makes the description of
Polymath as ``massive online collaboration'' somewhat excessive when it
is purely judged by the numbers. This isn't really surprising. The scale
of collaboration within mathematics is generally smaller than in other
sciences (see \cite{grossman2002patterns} and Tao's
\href{https://terrytao.files.wordpress.com/2015/07/polymath.pdf}{presentation}
to the 2014 \emph{Breakthrough Mathematics Symposium}), and this
substantially lowers the threshold for what we may consider large, or
even massive, collaborations in mathematics. More importantly, Polymath
does not fit the mantra of citizen science that ``anyone can be a
researcher .'' Even though Polymath is open to amateur mathematicians,
and the choice for problems in combinatorics is at least partly
motivated by the fact that knowledge of a small number of techniques can
be sufficient to tackle open problems\textbf{(ref)}, active contribution
is limited to those with the required expertise (the vertical axis of
the taxonomy of crowd science in Figure 4 in Franzoni \& Sauermann
2014). Polymath is meant to succeed in virtue of the fact that
``different people know different things, so the knowledge that a large
group can bring to bear on a problem is significantly greater than the
knowledge that one or two individuals will have'' (Gowers' initial
post), or by leveraging so-called micro-expertise
\cite{Nielsen:ReinventingDiscoveryTheNewEraOf:2012}. One may therefore
hesitate to see this type of approach as one that relies on the
\emph{Wisdom of Crowds}. This last consideration immediately emphasises
that it remains unclear what exactly the Polymath-projects are
crowdsourcing. At this point, we can only indicate that the intended
individual contributions are neither complete proofs (cfr. the \emph{one
correct answer} that is typically the goal on the StackExchange sites
--- \textbf{note: how is this on MathOverflow?}) nor clearly
identifyable or even just potential fragments of such a proof.

Thirdly (and not entirely disjoint from the question of what is being
crowd-sourced or aggregated), in contrast to both Zooniverse-like
endeavours and large-scale collaborations in the sciences, the
problem-statement in Polymath-projects is clear and explicitly stated,
but individual tasks are rarely clearly individuated and tasks are never
explicitly attributed to specific participants. Mainstream
citizen-science is typically based on a huge amount of similar tasks
that are explicitly attributed to participants in such a way that all
tasks are not only completed, but are completed multiple times to allow
for error-correction \textbf{(need reference)}. Similarly, outside the
scope of citizen-science, e-research and other large collaborative
enterprises exhibit ``a high degree of mutual dependence and task
certainty'' (\cite{meyer2015knowledge} building on work of Whitley 2000)
that has no clear counterpart in the Polymath projects. This leads
Franzoni \& Sauermann 2014 to classify the Polymath projects as
\emph{ill structured} projects that are characterised by the absence of
independent sub-tasks (the horizontal axis of their above-mentioned
Figure 4).

    \textbf{we will return to this question} One way to think of what PM is
crowdsourcing is in terms of finding the right people with the right
expertise (though what this expertise might be is unknown at the start),
but this is in general a problem for crowd science: link the right
project to the right people.

\textbf{but the task is very different, and divide and conquer methods
are not so easily applicable, which suggests that Polymath at the scale
of zooniverse is probably impossible. Need for expertise!!}

\begin{itemize}
\tightlist
\item
  The problem-statement in Polymath-projects is clear and explicitly
  stated, but individual tasks are rarely clearly individuated
  (support?) and tasks are never explicitly attributed to specific
  participants. Here, the contrast with much of mainstream
  citizen-science couldn't be greater, since these are typically based
  on a huge amount of similar tasks that are explicitly attributed in
  such a way that all tasks are not completed, but are completed
  multiple times to allow for error-correction (need reference). Even
  outside the scope of citizen-science, e-research and other large
  collaborative enterprises exhibit ``a high degree of mutual dependence
  and taks certainty'' (Meyer and Schroeder 2015 building on work of
  Whitley 2000) that has no clear counterpart in the Polymath projects.
\item
  A corrolary of the former contrast is that in the case of Polymath it
  isn't always clear what exactly is being crowdsourced. Even though the
  intended goal is a completed proof, the individual contributions are
  not small pieces of this proof (Van Bendegem 2011). Though it requires
  additional evidence in its support, one could make the point that
  Polymath isn't primarily build on the possibility of attracting the
  required amount of effort to complete a series of well-defined tasks,
  but rather about creating the right kind of community that can
  interactively identify the various bits that make up the final proof
  (finding the right division of labour is itself part of the
  collaborative process).
\item
  Typical citizen-science has roughly the following pattern: the
  organisation decides on the individual tasks which are then
  distributed among the participants. Each participant supplies his or
  her results back to the organisers, who then aggregate the results to
  answer the broader question they wish to answer. In other words,
  participants only answer small questions, and organisers exclusively
  deal with big and small questions, and finally also provide the big
  answer. This is not so in the case of Polymath, where questions and
  answers (small and big) go in both directions, and the only exclusive
  task left to the organisers is related to intermediate summaries that
  progressively track progress and try to (re)direct attention to
  specific problems.
\end{itemize}

Some of these features are no doubt associated with the nature of
mathematical inquiry, but they can also be generalised. One crucial
aspect is the importance of heuristics in mathematical inquiry. Even
when there is a consensus on what counts as accepted methods, the gap
between finding the right way to tackle a problem and actually solving
the problem is typically small. A less obvious, but maybe even more
influential aspect is that Polymath is itself an example of small-scale
science; both in terms of the size of the problems and in terms of the
budgetary resources and requirements (the contrast-case is Big Science,
which clearly intersects with some wellknown citizen-science projects).

** one of the crucial reasons of why Polymath qualifies as e-research is
exactly also why it is so appealing to the philosophy of mathematical
practice. Yet, this also gives rise to a paradox: to study the practice
we need to study a case where that practice is at least in part called
into question**

    \subsection{The role of ICT's}\label{the-role-of-icts}

The role of ICT in mathematical practices is in the first place tied to
the use of software artefacts to complete mathematical tasks like
numerical computation or mathematical exploration, or more proof-related
tasks like checking cases, checking entire proofs, or even constructing
proofs {[}need references to the literature{]}. Completing such tasks
with the help of computers typically requires one to make the
mathematics more accessible to computers. In some cases, like numerical
computation, this isn't really an issue. When it comes to actual proofs,
this becomes a huge challenge as the gap between proofs that are
intelligible for a human (or can be constructed using heuristics that
are efficient given human cognitive capacities) and proofs that can be
checked by methods that can be implemented in software remains large.
One way to look at this use of ICT's in mathematics is that it changes
the practice by modifying what we mean by a proof and thereby make
proofs accessible to computers-based methods. The computation, in such
cases, is outsourced to computers, but only after the mathematics was
adapted to make the tasks tractible. In a sense, the computers become
semantic engines, but with the proviso that the material on which they
operate is modified to suit their capabilities, rather than that their
capabilities are modified to suit the material on which they should
operate. The use of ICT in Polymath is of a wholly different nature, as
it remains a human-centric practice, and software platforms are only
used to connect multiple human actors. Using a contrast suggested in
Floridi 2009, where traditional uses of ICT's in mathematics have to
follow the path suggested by the semantic web, Polymath operates within
the bounds of the Web 2.0 paradigm by relying on available resources to
produce mathematical proofs, and using online platforms to allow them to
interact and aggregate their insights and capabilities.

    \section{A network-analysis of the
Polymath-projects}\label{a-network-analysis-of-the-polymath-projects}

\textbf{TO DO: what is network-analysis and why does it matter?}

\textbf{basic attitude: network-analysis forms the backbone of our
data-analysis because, first, comment-threads already have a
tree-structure that suggests a first network-like interaction, second,
the interactions between different partcipants can be modelled as
networks, and finally because even the relations between higher-level
entities like threads and projects can be studied by modelling them as
networks}

The blog-discussions that make up the different Polymath-projects were
analysed using the following techniques.

\begin{enumerate}
\def\labelenumi{\arabic{enumi}.}
\item
  Given a list of the different comment-threads for each project (each
  project comprises between 3 and 37 different comment-threads), the
  html-code of each thread was parsed to obtain the underlying
  tree-structure of each comment-thread, encoded as a directed graph
  with each comment as a separate node, directed edges as links between
  child and parent comments, and node-attributes with relevant meta-data
  for each comment (author, word-count, the actual content, time-stamp,
  \emph{etc.}).
\item
  Using the timestamps attributed to each comment, episodes in each
  discussion-threads were extracted using the MeanShift algorithm. Each
  episode is a period of intense commenting that appears to belong to
  the same temporal episode.
\item
  For each comment-thread or set of comment-threads (e.g.~all threads
  from the same project), a directed graph with the following features
  was created:

  \begin{itemize}
  \tightlist
  \item
    A single node for each participant to these threads.
  \item
    Weighted directed edges between nodes such there is a directed edge
    with weight \(n\) between participant A and B if and only if there
    are \(n\) directed edges between comment-nodes with A as author and
    comment-nodes with B as author.
  \end{itemize}
\end{enumerate}

The resulting graph is an interaction-network, for each edge signals
that two participants directly interacted at some point.

\begin{enumerate}
\def\labelenumi{\arabic{enumi}.}
\setcounter{enumi}{2}
\item
  For each comment-thread or set of comment-threads, a graph with the
  following features was created:

  \begin{itemize}
  \tightlist
  \item
    A single node for each participant to these threads.
  \item
    Weighted undirected edges between nodes such that there is an edge
    with weight \(n\) between participant A and B if and only if there
    are \(n\) episodes in the discussion-threads in which both A and B
    commented.
  \end{itemize}
\end{enumerate}

Additionally, information about all the threads (size, participants,
blog, type of thread, \emph{etc}) was assembled to allow high-level
comparisons between projects and between threads. At this stage the
names of all participants were compared (using
\texttt{difflib.SequenceMatcher} to check for string-similarity) to find
similar names. Participants with clearly similar names were checked, and
(when warranted, e.g.~in the case of typos or slightly different ways of
writing the same name) renamed to make the identification of
contributors across all projects possible.

All of the analysis was carried out using existing research libraries
for Python. In particular BeautifullSoup for html-parsing, networkx
(\cite{hagberg-2008-exploring}) for network-analysis, pandas
(\cite{mckinney-proc-scipy-2010}) for data-analysis, matplotlib
(\cite{Hunter2007}) for most visualisations, scikit-learn
(\cite{scikit-learn}) for the above-mentioned MeanShift
clustering-algorithm, and Scipy (\cite{scipy}) for the hierarchical
linkage used in the next section.

\textbf{A note on the scope of our data:} The selection of the
discussion-threads was in the first place based on the lists provided by
the
\href{http://www.michaelnielsen.org/polymath1/index.php?title=Main_Page}{Polymath
Wiki}. In the case of Polymath 3 we extended this with 2 more threads
that were clearly part of the development of this project. In other
cases we left out related blog-posts and discussion-threads that were
not listed on the wiki. Typical cases of the material we ignored are
blog-posts written before the actual start of a project, and devoted to
logistics or to the description of background-material.

\textbf{TO DO}: describe the different levels of analysis!

    \section{Polymath as a scientific
community}\label{polymath-as-a-scientific-community}

    One possible answer to the question of what the Polymath-projects are
crowdsourcing, is the community that is needed to solve a given problem
rather than the individual pieces of the solution itself. (cfr. the
remark that after the fact it was obvious who should have been involved,
but this couldn't be known in advance \textbf{ref}) Irrespectively of
whether this is indeed the best conceptual framework to analyse Polymath
\textbf{(an alternative would be to think of it as a platform; like
Zooniverse)}, this possibility suggests that we should take a closer
look at the structure and temporal evolution of the community of
participants to all Polymath-projects.

\textbf{digression} In general, most platform-based collaborations are
about assigning projects/tasks to potential participants. I.e. they are
all in the business of crowdsourcing a community. The question we ask
here is whether Polymath as a whole is more community-like than
platform-like. This is best done by trying to find out if we can
identify a community across projects rather than just within individual
projects (assuming there is such a community within individual
projects). \textbf{end of digression}

Our previous overview already revealed a substantial variation of the
number of participants in each project. By looking at who leaves and who
joins after each project, we see that changes in the community is both
due to new participants joining, and participants not joining the next
project after taking part in the previous one (remark: this claim is
only partially true, since for instance Polymath 10 and 11 are
simultaneously active).


    \begin{figure}
        \begin{center}\adjustimage{max size={0.9\linewidth}{0.4\paperheight}}{pmagain_files/pmagain_36_0.png}\end{center}
        \caption{}
        \label{}
    \end{figure}
    
    \begin{figure}
        \begin{center}\adjustimage{max size={0.9\linewidth}{0.4\paperheight}}{pmagain_files/pmagain_36_1.png}\end{center}
        \caption{}
        \label{}
    \end{figure}
    
    If we restrict our attention to participants that were active in at
least 2 threads (less than 14\% of all participants), we can start to
look for patterns in how the core contributor-base evolves. A first
purely qualitative representation of who participated to which projects
shows both that as a long-term community Polymath is extremely small,
and that the basis for the majority of this small core was largely
constituted in the early stages of Polymath.


    \begin{figure}
        \begin{center}\adjustimage{max size={0.9\linewidth}{0.4\paperheight}}{pmagain_files/pmagain_38_0.png}\end{center}
        \caption{}
        \label{}
    \end{figure}
    

    \begin{figure}
        \begin{center}\adjustimage{max size={0.9\linewidth}{0.4\paperheight}}{pmagain_files/pmagain_39_0.png}\end{center}
        \caption{}
        \label{}
    \end{figure}
    
    To refine these insights we considered the number of comments in each
project for the same set of participants. Using hierarchical linkage
based on the Euclidean distance between vectors (number of comments per
project per contributor and vice-versa), we identified participants and
projects with similar patterns of participation. The contributors and
projects were reordered on the basis of these results to reveal patterns
in the heatmap included below.

(Note that a log10-scale is used, which compresses the higher values and
improves visibility of small differences in the lower values. The
color-scale is chosen such as to improve the visibility of low values.)


    \begin{figure}
        \begin{center}\adjustimage{max size={0.9\linewidth}{0.4\paperheight}}{pmagain_files/pmagain_41_0.png}\end{center}
        \caption{}
        \label{}
    \end{figure}
    
    This approach suggests natural grouping of the participants (recall,
this is only about a small subset of all participants) based on
similarity:

\begin{itemize}
\tightlist
\item
  \textbf{Leaders (2):} Tao \& Gowers
\item
  \textbf{Core-participants (11):} Peake -\textgreater{} Sauvaget
\item
  \textbf{Periphery (49):} Kowalski -\textgreater{} mixedmath
\item
  \textbf{Outsider (1):} Edgington (very active in PM5 and PM11, but not
  an early adopter)
\end{itemize}


    \begin{figure}
        \begin{center}\adjustimage{max size={0.9\linewidth}{0.4\paperheight}}{pmagain_files/pmagain_43_0.png}\end{center}
        \caption{}
        \label{}
    \end{figure}
    
    The grouping that arises from restricting our attention to
research-threads is slightly different:

\begin{itemize}
\tightlist
\item
  \textbf{Leaders (2):} Tao \& Gowers
\item
  \textbf{Core-participants (8):} Ryan O'Donnel -\textgreater{} Sune
  Kristian Jakobsen\\
\item
  \textbf{Periphery (53):} Croot -\textgreater{} Malkevitch
\end{itemize}

One general conclusion holds for both representations, namely that the
group of core participants is rather heterogeneous. Indeed more so than
the participants in the periphery. \textbf{(note: isn't this to be
expected that those who participated to a larger number of projects have
a higher chance to be less similar (given the metric we use)}

\textbf{More importantly:} What is considered as the periphery across
all projects includes some central figures of individual projects. We
discuss this explicitly w.r.t. Polymath 8, but it is also true for
Polymaths 4 and 5 (e.g.~Ernie Croot is one of the most prominent
participants to Polymath 4).

    The most prominent projects according to these data are Polymath 1, 5,
and to a lesser extent 4. The largest project, namely Polymath 8, is
less prominent, and clearly includes fewer highly active participants
from the core community of participants we identified. This suggests
that limiting our attention to participants to at least two projects
leaves out important contributors to Polymath 8.

A scatter-plot of the same participation-data for all participants
allows us to expand our scope without sacrificing legibility.





    \begin{figure}
        \begin{center}\adjustimage{max size={0.9\linewidth}{0.4\paperheight}}{pmagain_files/pmagain_49_0.png}\end{center}
        \caption{}
        \label{}
    \end{figure}
    
    The above figure displays for each author the number of projects (s)he
participated to (X-axis) and the avergage number of comments made to
each of these projects (excluding the projects with null participation).
On a total of 465 identified (non-anonymous) participants, 402
contributed to only 1 project, 32 to 2 projects, 16 to 3 projects, 10 to
4 projects, 2 to 6 projects, 1 to 8 projects, and 2 to 9 projects.

If we exclude the core-participants, there does not appear to be any
marked correleation between active participation to projects and
continued participation. Additionally, we can identify a group of 7
participants that were extremely active (more than 150 comments) during
only one project. Further inspection of the data revealed that 6 of
these participated to Polymath 8. These data-points partially explain
why the largest project is under-represented in the data that reveal
overlaps between projects.

    Now we limit this

    \subsubsection{TO DO}\label{to-do}

\begin{itemize}
\tightlist
\item
  check if mean takes into account projects with zero participation
  (current version averages over threads with actual participation by
  replacing 0 comments by NaN)
\item
  ``by thread'' would give more balanced result since participation to
  large projects would give higher number of comments
\item
  consider ``participated in at least n projects'' as an alternative
\item
  try to use box-plots as well (per number of participations)
\end{itemize}



    \begin{figure}
        \begin{center}\adjustimage{max size={0.9\linewidth}{0.4\paperheight}}{pmagain_files/pmagain_54_0.png}\end{center}
        \caption{}
        \label{}
    \end{figure}
    
    \begin{figure}
        \begin{center}\adjustimage{max size={0.9\linewidth}{0.4\paperheight}}{pmagain_files/pmagain_54_1.png}\end{center}
        \caption{}
        \label{}
    \end{figure}
    

            \begin{Verbatim}[commandchars=\\\{\}]
{\color{outcolor}Out[{\color{outcolor}34}]:} \{'Polymath 1': 0.9777777777777777,
          'Polymath 10': 0.9777777777777777,
          'Polymath 11': 0.9777777777777777,
          'Polymath 2': 1.0,
          'Polymath 3': 0.9777777777777777,
          'Polymath 4': 0.9777777777777777,
          'Polymath 5': 0.9777777777777777,
          'Polymath 6': 0.9777777777777777,
          'Polymath 7': 1.0,
          'Polymath 8': 0.9777777777777777,
          'Polymath 9': 0.9777777777777777\}
\end{Verbatim}
        
    At the same time, because Polymath 8 is almost twice as large as the
second largest project, these higher averages should not be taken as an
indication that Polymath 8 participants were more active participants. A
focus on the level of threads can reduce the size-effect of larger
projects, since the variation in the size of threads (mean of 80 and std
of 52, which gives a coeffient of variation of 0.65) is smaller than the
variation in the size of projects (mean of 993 and std of 1175, which
gives a coefficient of variation of 1.18). \textbf{Is the variation then
really smaller? maybe it's just ``above the mean'' that projects show
less variation\ldots{}}




    \begin{figure}
        \begin{center}\adjustimage{max size={0.9\linewidth}{0.4\paperheight}}{pmagain_files/pmagain_59_0.png}\end{center}
        \caption{}
        \label{}
    \end{figure}
    
    The above scatter-plot displays average comments against participation,
but now at the finer level of comment-threads (for a total of 137
comment-threads and the same 465 identified participants). Here too the
core participants stand out. If we additionally take into account that
the threads have on average 80 comments (min:1, 25\% quantile: 37, 50\%
quantile: 91, 75\% quantile: 108, max: 416), the dominant presence of
Gowers and Tao can hardly be missed. Since we focus on threads instead
of on projects, larger projects like Polymath 8 have, at this level of
analysis, a less marked effect. Overall, the area with up to 5 comments
per thread and up to 10 threads is the most populated with 403 out of
465 participants (87\%; approximately the same as the number of
participants that took part in only one project). Participation to only
a few threads does not exclude active participation, but participation
to more threads makes low participation less likely. A closer
examination of the data confirms that the top-left part of the plot is
no longer restricted to participants to Polymath 8, as we found for the
project-level data.

    \textbf{ToDo} Do paths of more than 1 really matter? (in general few
components seem to have a large diameter)



    \begin{Verbatim}[commandchars=\\\{\}]
60 weakly connected components
0 5
1 0
2 0
3 0
4 0
5 0
6 0
7 0
8 0
9 0
10 0
11 0
12 0
13 0
14 0
15 0
16 0
17 0
18 0
19 0
20 0
21 0
22 0
23 0
24 0
25 0
26 1
1
27 1
1
28 0
29 0
30 0
31 0
32 0
33 0
34 0
35 0
36 0
37 0
38 0
39 0
40 0
41 0
42 0
43 0
44 0
45 0
46 0
47 0
48 0
49 0
50 0
51 0
52 0
53 0
54 0
55 0
56 0
57 0
58 0
59 0

    \end{Verbatim}

    \begin{figure}
        \begin{center}\adjustimage{max size={0.9\linewidth}{0.4\paperheight}}{pmagain_files/pmagain_63_1.png}\end{center}
        \caption{}
        \label{}
    \end{figure}
    

    \begin{Verbatim}[commandchars=\\\{\}]

        ---------------------------------------------------------------------------

        NetworkXError                             Traceback (most recent call last)

        <ipython-input-40-57b3eb271a3c> in <module>()
    ----> 1 nx.diameter(the\_graphs[0].to\_undirected())
    

        /Users/patrickallo/anaconda3/lib/python3.5/site-packages/networkx/algorithms/distance\_measures.py in diameter(G, e)
         94     """
         95     if e is None:
    ---> 96         e=eccentricity(G)
         97     return max(e.values())
         98 


        /Users/patrickallo/anaconda3/lib/python3.5/site-packages/networkx/algorithms/distance\_measures.py in eccentricity(G, v, sp)
         61         if L != order:
         62             msg = "Graph not connected: infinite path length"
    ---> 63             raise networkx.NetworkXError(msg)
         64 
         65         e[n]=max(length.values())


        NetworkXError: Graph not connected: infinite path length

    \end{Verbatim}






    \paragraph{Food for thought}\label{food-for-thought}

\begin{itemize}
\tightlist
\item
  Average length of comments vs number of comments (author and threads).
\item
  plot in/out-degree against number of comments (do one-time commenters
  get responses) (use \texttt{scatter\_matrix})
\item
  See if discussion happens at more than one place in thread. Do edges
  in tree cross?
\end{itemize}

    \section{Investigating individual
projects}\label{investigating-individual-projects}

    Projects can be studied from the following perspectives:

\begin{enumerate}
\def\labelenumi{\arabic{enumi}.}
\tightlist
\item
  General features of a project, like the speed at which it grows, or
  the distribution of commenting-activity between the different threads.
\item
  A network-perspective on the relevant contributors, and either taking
  into account direct interactions, or co-presence in episodes (cfr.
  Section 4).
\item
  A quantitative perspective based on the number of contributions of
  each participant, possibly split up between threads, or between levels
  of comments.
\end{enumerate}

We illustrate these by reconsidering Polymath 1.




    \textbf{TODO} need to tweak parameters to get drawing of network more
pleasant (more in the centre)








    \subsubsection{Topics to be addressed}\label{topics-to-be-addressed}

\begin{itemize}
\item
  the role of hierarchical comments
\item
  the parellel development in PM1
\item
  do we see a change if we compare the different succesful projects
\item
  how do projects ``die''
\item
  what's the best model to understand the role of the central figures?

  \begin{itemize}
  \tightlist
  \item
    more connected
  \item
    more active
  \item
    higher proportion of higher-level comments (they do reply more,
    sometimes even systematically)
  \end{itemize}
\item
  the difference between research and discussion threads
\item
  what's the difference between PM1, PM4 and PM8 (the clearly succesfil
  projects)
\item
  what's the difference between equally large projects with different
  degrees of success (PM4 vs PM5)
\end{itemize}


    \subsection{References}\label{references}

\bibliography{pmagain_files/pmagain}

    \subsection{Appendix}\label{appendix}

\subsubsection{Overview of the threads}\label{overview-of-the-threads}

\paragraph{Research-threads}\label{research-threads}


    \paragraph{Discussion-threads}\label{discussion-threads}


    \subsubsection{Identification of
authors}\label{identification-of-authors}

The approach that was taken probably erred on the cautious side.
\textbf{(Example: can we assume that HH, Harald and Harald Helfgott are
the same?)}




    % Add a bibliography block to the postdoc
    
    
    
    \end{document}
